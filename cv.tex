%% start of file `cv.tex' adapted from Moderncv 'template.tex'.
%% Copyright 2006-2013 Xavier Danaux (xdanaux@gmail.com).
%
% This work may be distributed and/or modified under the
% conditions of the LaTeX Project Public License version 1.3c,
% available at http://www.latex-project.org/lppl/.

\documentclass[12pt,a4paper,sans]{moderncv}            % possible options include font size ('10pt', '11pt' and '12pt'), paper size ('a4paper', 'letterpaper', 'a5paper', 'legalpaper', 'executivepaper' and 'landscape') and font family ('sans' and 'roman')

% moderncv themes
\moderncvstyle{classic}                                % style options are 'casual' (default), 'classic', 'oldstyle' and 'banking'
\moderncvcolor{blue}                                   % color options 'blue' (default), 'orange', 'green', 'red', 'purple', 'grey' and 'black'
%\renewcommand{\familydefault}{\sfdefault}             % to set the default font; use '\sfdefault' for the default sans serif font, '\rmdefault' for the default roman one, or any tex font name
%\nopagenumbers{}                                      % uncomment to suppress automatic page numbering for CVs longer than one page

% character encoding
\usepackage[utf8]{inputenc}                            % if you are not using xelatex ou lualatex, replace by the encoding you are using

% adjust the page margins
\usepackage[scale=0.75]{geometry}
% \setlength{\hintscolumnwidth}{3cm}                   % if you want to change the width of the column with the dates
% \setlength{\makecvtitlenamewidth}{10cm}              % for the 'classic' style, if you want to force the width allocated to your name and avoid line breaks. be careful though, the length is normally calculated to avoid any overlap with your personal info; use this at your own typographical risks...

\newcommand\Colorhref[3][color1]{\href{#2}{\small\color{#1}#3}}

% personal data
\name{Robert}{Maerten}
\title{Administrateur Systèmes}                        % optional, remove / comment the line if not wanted
\address{<adresse masquée>}{<ville masquée>}  % optional, remove / comment the line if not wanted; the "postcode city" and "country" arguments can be omitted or provided empty
\phone[mobile]{<numéro masqué>}                         % optional, remove / comment the line if not wanted; the optional "type" of the phone can be "mobile" (default), "fixed" or "fax"
% \phone[fixed]{+2~(345)~678~901}
% \phone[fax]{+3~(456)~789~012}
\email{bob.maerten@gmail.com}                          % optional, remove / comment the line if not wanted
\homepage{bobmaerten.github.io}                 % optional, remove / comment the line if not wanted
% \social[linkedin]{john.doe}                          % optional, remove / comment the line if not wanted
% \social[twitter]{bobmaerten}                         % optional, remove / comment the line if not wanted
\social[github]{bobmaerten}                            % optional, remove / comment the line if not wanted
% \extrainfo{additional information}                   % optional, remove / comment the line if not wanted
% \photo[64pt][0.4pt]{picture}                         % optional, remove / comment the line if not wanted; '64pt' is the height the picture must be resized to, 0.4pt is the thickness of the frame around it (put it to 0pt for no frame) and 'picture' is the name of the picture file
% \quote{Some quote}                                   % optional, remove / comment the line if not wanted
%----------------------------------------------------------------------------------
%            content
%----------------------------------------------------------------------------------
\begin{document}
\makecvtitle

\section{Expériences}
\cventry{Depuis 2011}{Administrateur Systèmes}{Université de Lille3}{Villeneuve d'ascq}{}{%
\begin{itemize}
\item Migration d'une infrastructure de serveurs physiques/virtuels(Xen) sur VMWare.
\item Regroupement des différents outils de supervision sur une plate-forme Centreon.
\item Mise en place de la centralisation des configurations, automatisée avec Puppet.
\end{itemize}}

\cventry{2008--2010}{Responsable de Projets}{Université de Lille3}{Villeneuve d'ascq}{}{Coordinateur de la cellule d'assistance à la maitrise d'ouvrage pour les projets.
\begin{itemize}%
  \item Étude sur les référentiels des personnes, les sources et les flux d'alimentation.
  \item Pilotage du développement de la plate-forme de gestion de la formation continue.
  \item Développement itératif et incrémental d'une plate-forme d'auto-apprentissage en langues, s'appuyant sur une méthode agile adaptée de Scrum.
\end{itemize}}

\cventry{2005--2008}{Responsable des Systèmes d'information}{Délégation CNRS NPdC}{Lille}{}{Chef du service informatique de la délégation.
\begin{itemize}%
  \item Coordination de la politique de sécurité des systèmes d'information en région.
  \item Administration des systèmes décentralisés de la DSI pour les laboratoires.
  \item Gestion du budget, relations avec les partenaires et les fournisseurs.
\end{itemize}}

\cventry{2003--2005}{Administrateur Systèmes}{Université de Lille3}{Villeneuve d'ascq}{}{Webmaster du site \href{http://www.univ-lille3.fr}{www.univ-lille3.fr}.
\begin{itemize}%
  \item Migration de l'infrastructure VAX/VMS sur des serveurs GNU/Linux (Debian).
  \item Développement en PHP du portail de l'Université et des services web associés.
\end{itemize}}

\cventry{1999 - 2003}{Gestionnaire de Parc Informatique}{Université de Lille3}{Villeneuve d'ascq}{}{%
\begin{itemize}%
  \item Mise en réseau d'un parc de machines Windows via un serveur Linux/Samba.
\end{itemize}}

\cventry{1996 - 1998}{Assistant Ingénieur}{CERIM - Université de Lille2}{Lille}{}{%
\begin{itemize}%
  \item Support informatique aux chercheurs du laboratoire et administration des systèmes.
\end{itemize}}

\section{Formation Initiale}
\cventry{1994--1996}{Suivi des enseignements Licence/Maitrise informatique}{USTL}{Lille}{}{}  % arguments 3 to 6 can be left empty
\cventry{1992--1994}{BTS informatique de gestion}{Lycée G. Berger}{Lille}{Mention Bien}{}
\cventry{1992}{Baccalauréat C}{Lycée Paul Duez}{Cambrai}{}{}
\section{Formation Continue}
\cventry{Mars 2012}{Bilan de compétences}{SUDES}{Villeneuve d'ascq}{}{Entretiens et travail sur la valorisation des expériences en vue d'un recadrage professionnel.}
\cventry{Mars 2011}{Mac OS X Support and Server Essentials 10.6}{Agnosys}{Villeneuve d'ascq}{Formation administration système}{2 semaines de formation à la prise en main des versions client et serveur de Max OS X 10.6. Obtention de « Apple Certified Support Professionnal 10.6 ».}
\cventry{Septembre 2010}{Atelier Git Attitude}{Ciblo/Clever Age}{Paris}{Formation développeur}{Découverte de Git afin de gagner en productivité dans le domaine de la gestion des versions.}
\cventry{Mai 2010}{Maîtriser la gestion des processus métiers}{Infhotep}{Paris}{Formation au BPM-N}{Mise en œuvre des étapes d’un projet de Business Process Management depuis la définition du périmètre jusqu’à l’optimisation des processus.}
\cventry{Mai 2007}{Fondamentaux du Référentiel ITIL}{Interne CNRS}{Paris}{}{Acquisition du vocabulaire et des enjeux du référentiel ITIL\@. Connaissances et compétences pour décrire les disciplines, les processus, les rôles et les fonctions dans l'organisation.}
\cventry{Juillet 2005}{UML pour le recueil des besoins}{Interne CNRS}{Paris}{}{Étude des différents diagrammes proposés par UML afin de les utiliser dans les phases de recueil de besoins.}

\section{Compétences}
\cvitem{Administration Systèmes}{GNU/Linux (Debian, Ubuntu, RedHat), Windows/AD (notions), Puppet, Time Navigator}
\cvitem{Virtualisation}{VMWare, Xen, Vagrant, AWS (notions).}
\cvitem{Supervision}{Munin, Nagios, Centreon}
\cvitem{Développement}{Ruby, Rails, PHP, JavaScript(notions), outils GNU, SVN, Git.}
\cvitem{Gestion de projets}{Gantt, Modélisation BPM, Méthodes agiles (Scrum, Kanban),\newline{}\emph{Test Driven Development}.}
\cvitem{Plate-forme}{GitHub, GitLab, Trac, GLPI, Moodle.}
\cvitem{Bureautique}{OpenOffice, GoogleApps, \LaTeX, Gantt|MS Project, BizAgi Modeler.}
\cvlanguage{Anglais}{Lu, écrit, parlé.}{}

\section{Centres d'intérêt}
\cvitem{Loisirs}{Roller \emph{freeride}. Trotinette/Vélo pour les déplacements urbains.\newline{}Patisserie amateur.}
\cvitem{Associatif}{Membre organisateur de \Colorhref{http://takeoffconf.com}{Takeoff Conf} et des prochains Takeoff Talks à Lille.}

\end{document}
